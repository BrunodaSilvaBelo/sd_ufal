\documentclass[conference]{IEEEtran}
\usepackage[utf8]{inputenc}
\usepackage[brazilian]{babel}
\usepackage{indentfirst}
\usepackage[super, sort]{natbib}
\bibliographystyle{IEEEtranN}

\title{Efficient execution of web navigation sequences \cite{art:main}
  \\ A review}

\author{
  \IEEEauthorblockA{Universidade Federal de Alagoas -- Instituto de Computação
    \\ Professor: André Lage Freitas -- Matéria: Sistema Distribuídos}
  \IEEEauthorblockN{Revisado por: \textbf{Bruno da Silva Belo} (12110981)
    e \textbf{Iago Barbosa de Souza} (14210353) em 2016-08-13}
}
\begin{document}
\maketitle

\section{Introdução}
\label{sec:intro}
Em aplicações webs, cada parte da aplicação é divididas em DOM (document object
model), e sua navegação é feita através de eventos, \emph{listener}, destes DOM.
Assim para testar tais aplicações é necessário simular uma interação humana com
a página, um jeito simples de fazer isto é utilizando a forma padrão para
navegá-la, assim carregando a aplicação via \emph{html} e algum outro programa
para fazer a sequência de interação com ela. Porém este jeito a uma desvantagem,
ela carregará tudo da aplicação, assim parte dela que não é usada na sequência é
carregado desnecessariamente, logo ocupando espaço de memória e tempo de
processamento que será inútil, pois eles não serão usado. Assim a proposta do
artigo é alcançar um modo otimizado para executar estas sequência a qual só será
carregado e usado aquilo que realmente é preciso para executar tal interação.
\section{Contribuição}
\label{sec:contrib}
Os autores criaram um navegador próprio cujo nome é exp-IE e é baseado no
Microsoft Internet Explorer,  para poder otimizar a sequência de navegação, isto
é, como um arquivo \emph{html} é representado como uma árvore nos browser, eles
conseguiram fazer com que dado um ação em um nó da árvore quais outros nós ou
sub-árvores são relevante ou irrelevante para tal sequência de ações, assim
estes nós nem seriam baixados e carregados, mas aqueles sim.

Antes, este problema era resolvido utilizando navegadores já existentes, como
eles foram feito voltado para humanos, eles utilizam muito recursos do
computador, assim a criação de um próprio é uma boa abordagem, como é mostrado
no artigo a qual o exp-IE otimizado chegou a ser 6x mais rápido que estes outros
navegadores e é mostrado que além de melhoria de performance existe também
diminuição do uso de memória .
\section{Discussão}
\label{sec:discussao}
O artigo\cite{art:main} é consideravelmente simples de ler, não possui muitos
termos técnicos, logo até quem não sabe muito ou é interessa por desenvolvimento
web vai conseguir entender fácil o que é dito nele, como ele explica alguns
assuntos de web, por exemplo o funcionamento do html em browser, explica um
pouco sobre DOM, este artigo também é bom para quem é curioso na área.

A técnica apresentada pode ser importante para quem desenvolve tecnologias web,
principalmente para parte de testes, já que o tempo de teste das aplicações pode
ser diminuído consideravelmente, a média obtida foi até 3x mais rápido, logo
para aplicações muito grandes, há vantagem de utilizá-la.

O problema de \emph{web navigation sequences} já foi debatido, porém o
diferencial deste é o fato de fazê-lo carregando somente o necessário e os
autores falam que ele também é bastante resiliente, isto é, é possível que a
mesma solução seja aplicado caso aconteça pequenas mudanças na aplicação, como
mudança na interface.

\textbf{Apresentação} Apesar de ter identificado um erro gramatical, escreveram
\emph{Experimient} no lugar de \emph{Experiment}, possui uma leitura
relativamente fácil pelo motivo que já foi dito, não possui muitos termos
técnicos.

\textbf{Originalidade} Sequência de navegação já foi debatida, porém neste
artigo\cite{art:main} utilizaram a abordagem de carregar somente o que era
necessário e a criação de um navegador próprio, voltado para máquina, para isto.
Assim pode-se dizer que é uma boa contribuição para área.

\textbf{Relevância} A abrangência deste artigo não passa do seu domínio, isto é,
aplicações web, porém para a área pode ser uma boa ferramenta saber desta
técnica.

\textbf{Geral} Para envolvidos e interessados neste área é um artigo muito bom
para se ler, mas ele não consegue sair muito do seu domínio, pois este é um
problema muito especifico para a área web. Sendo assim, para quem está de fora
da área, pode não ser importante.

\begin{table}[h]
  \centering
  \begin{tabular}{|c|c|c|c|c|c|}
    \hline
    \textbf{Critério} & \textbf{Ruim} & \textbf{Razoável} & \textbf{Médio}
    & \textbf{Bom} & \textbf{Excelente} \\
    \hline
    Apresentação & & & & x & \\
    \hline
    Originalidade & & & & x & \\
    \hline
    Relevância & & x & & & \\
    \hline
    Geral & & & x & & \\
    \hline
  \end{tabular}
  \caption{Avaliação do artigo\cite{art:main}}
  \label{tab:avaliacao}
\end{table}

\bibliography{revisao}
\end{document}